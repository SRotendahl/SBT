In practice, arbitrage is the process of buying something at price $x$
and instantly selling it a price $y$ where $x<y$. Arbitrage is
risk-free, if we make some assumptions; since we perform it instantly
we can guarantee that $x<y$, and we do not take into account the
inherit vulnerabilities of bugs in smart contracts.

\subsection{LOB arbitrage}

When determining the arbitrage opportunity it is a bit more
complicated since prices of assets are not static but determined by
market behaviors. As we defined in the background section there are
two main forms of exchanges, the simpler one being the limit order
book. When performing arbitrage on LOB DEX's we check if there is some
asset (ABC) that has an overlapping price, an example of this can be
seen in figure \ref{fig:ArbLOB}, here wee see that we can buy the ABC
asset at exchange Y and sell it at exchange X, and we can do that for
the overlapping limit orders.
\begin{figure}[h]
\centering
\includegraphics[width=0.7\textwidth]{assests/Flash-loans-Arbitrage-Overlap-1}
\caption{Arbitrage on LOB DEX}
\label{fig:ArbLOB}
\end{figure}
This is a simple form of arbitrage, we can find more opportunities if
we look at \textit{triangular arbitrage}. Triangular arbitrage is
where the arbitrage opportunity arises from the trade of 3 assets, that
is, we trade asset X for asset Y, trade asset Y for asset Z, and
finally trade asset Z back to asset X. We end up with the rule that if
$rate(X/Y)*rate(X/Z)>rate(Y/X)$ then there is a profitable arbitrage
(where $rate(A/B)$ is the conversion rate from $B$ to $A$). We have
visualized this with an example in figure \ref{fig:ArbTrig}.
\begin{figure}[h]
\centering
\includegraphics[width=0.4\textwidth]{assests/Flash-loans-Arbitrage-triangular}
\caption{Triangular arbitrage}
\label{fig:ArbTrig}
\end{figure}

\subsection{AMM arbitrage}
The process for detecting an arbitrage opportunity when working with AMMs, is
actually straight forward. Put simply, we just need to detect that one exchange, has
a different price than another\footnote{In our case, we have ignored fees, but
in a real system, this would of course need to be accounted for}. When we have
detected this, we need to buy at the cheaper, and sell at the more
expensive one. However, since the price at both exchanges will change
as a function of the amount of money we use, we can't just take a
flash loan as big as possible, and arbitrage
with the entire amount. At some point, the price has changed so much, that we're
no longer making money, but losing it. So we need to find the amount of
money, such that when all is said and done, the 2 exchanges will have the same
price. Why exactly the same price? Because if the first exchange is still
lower, there still exists an arbitrage opportunity, and we didn't fully exploit
the one we found. If the second exchange is at a
lower price than the first after the transaction is done, we will have
bought some of the second assets, at a
higher price, than we sold it for. While the entire transaction might
still be profitable, we didn't
maximize our profits, which we strive to do. So in order to maximize our
profits, we need to find out how much to exchange, such that the prices at the 2
exchanges are the same afterward. In the next section, we will show how to find
this amount.

\subsection{Maximizing AMM arbitrage opportunities}\label{maximizing}
As explained in the previous section, the prices at the 2
exchanges, involved in an arbitrage exploit, need to be the same after the
transaction, and that we need to find the input amount (amount to be
exchanged) required to
maximize our profits. In this section, we will find this amount. First, we will
state the equation that describes the desired outcome:

\begin{equation}
\overbrace{\frac{x_{1b} + input}{y_{1a}}}^{\text{The price at the
    first exchange, after the transaction}} = \overbrace{\frac{x_{2a}}{y_{2b} + (y_{1b} - y_{1a})}}^{\text{The price at the
    second exchange, after the transaction}}
\end{equation}

\begin{table}[h]
\centering
\begin{tabular}{|l|l|}
\hline
$x_{1b}$ & Amount of x, on the 1st exchange \textbf{b}efore the transaction \\
\hline
$y_{1b}$ & Amount of y, on the 1st exchange \textbf{b}efore the transaction \\
\hline
$y_{1a}$ & Amount of y, on the 1st exchange \textbf{a}fter the transaction \\
\hline
$x_{2b}$ & Amount of x, on the 2st exchange \textbf{b}efore the transaction \\
\hline
$x_{2a}$ & Amount of x, on the 2st exchange \textbf{a}fter the transaction \\
\hline
$y_{2b}$ & Amount of y, on the 2st exchange \textbf{b}efore the transaction \\
\hline
\end{tabular}
\end{table}

Given that we also know that the product of the reserves, are the same before and
after the transaction, we can present the following 2 equations that describe
that:

\begin{equation}
\underbrace{(x_{1b} + input)}_{\text{Reserve of asset x at the 1st exchange, after transaction}} \cdot y_{1a} = x_{1b} \cdot y_{1b}
\end{equation}

\begin{equation}
x_{2a} \cdot \underbrace{(y_{2b} + (y_{1b} - y_{1a}))}_{\text{Reserve of asset y at
the 2nd exchange, after the transaction}} = x_{2b} \cdot y_{2b}
\end{equation}

We know the values of $x_{1b}$, $y_{1b}$, $x_{2b}$ and $y_{2b}$, we don't know
the values of $input$, $y_{1a}$, and $x_{2a}$, and we would like to find the
value of $input$.

This is a simple system of 3 equations, with 3 unknowns, which can be solved
analytically. The final formula for input is a bit messy (see \ref{code:pyProf}
line 10), but we can use it to find the correct amount of input, in order to
maximize our profits.

\subsubsection{AMM arbitrage example}
Let's say we have 2 exchanges and the asset pair x/y. The first exchange having
100,000 $x$, and 10,000 $y$, the second exchange having 110,000 $x$, and 10,000 $y$.
Clearly, we have an arbitrage opportunity, since there will be a difference in
price. But how much should we exchange? If we enter these numbers into our
formula, we get that we should enter with $2,440.44 x$. We will then convert that
into $y$, and we will end up with:

\begin{equation}
intermediary\_y = 10,000 - \frac{100,000 \cdot 10,000}{100,000 + 2,440.44} = 238.23
\end{equation}

We will then convert it back at the second exchange, and we will end up with:
\begin{equation}
resulting\_x = 110,000 - \frac{110,000 \cdot 10,000}{10,000 + 238.23} = 2559.55
\end{equation}

We now need to take into account that we started with $2,440.44 x$, and we can
calculate our profits:
\begin{equation}
profit = 2,559.55 - 2,440.44 = 119.11
\end{equation}

