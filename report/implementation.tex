%! TEX root = main.tex
\subsection{Libraries}
\subsubsection{\code{money-legos}}
We chose to utilize the \code{money-legos} library. The advantage of using a
third party library is, that we have access to helper methods for many of the
different protocols. This made it easy to swap between doing flash loans at
different providers, and we were easily able to swap on different exchanges.
Furthermore the people behind \code{money-legos} have made some great
documentation for many of the protocols.

The \code{money-legos} library works by having a few helper methods, but mostly
by providing abstract\footnote{Pre Solidity 0.6.0 an abstract contract was
simply a contract without an implementation, but we chose to port it to solidity
0.6.0 to be able to use some new features} contracts that can then be
instantiated with the address of the deployed contract.

\subsubsection{\code{open-zeppelin}}

From \code{open-zeppelin} we have mostly taken inspiration from the code, but we
also directly inherit from the \code{Ownable} contract, as we don't want other
people to be able to use our contract.

\subsection{Flash loans on $\delta y/\delta x$ in practice}
In practice flash loans are implemented by chaining three method calls on the
$\delta y/\delta x$ protocol, where in other protocols flash loans are built
into the protocol.
\begin{enumerate}
    \item Borrow
    \item Call (execute money making logic)
    \item Deposit
\end{enumerate}
A small fee is paid at the deposit step (+2 wei of the borrowed asset).

\subsection{The smart contract}
Our contract named ``Gordon'' uses the $\delta y/\delta x$ protocol to take a
flash loan, and then based on the user input, execute arbitrage on different
DEXes. We chose to make our contract, so that you deploy it once, and based on
user input, when a flash loan is initiated it will chose to exchange on different
markets and different assets. This means that the contract can be reused and
does not need to be redeployed, for each arbitrage opportunity. This design
decision is critical due to two main reasons.
\begin{itemize}
    \item It is expensive to deploy a smart contract. Using the built in gas
        estimator in Remix IDE we estimated, that deploying the contract uses
        about 936,800 gas, which at low price of 77 Gwei equals around
        \$28 at the time of writing. We can excatly choose this low price, as
        our contract is deploy once, whereas otherwise, we would need to pay
        more per gas in order to reliably get our contract deployed quickly.
    \item If we have to wait for the contract to be deployed we might miss the
        arbitrage opportunity, as this would require us to wait for at least one
        block, while the contract is deployed, and then afterwards call the
        appropriate method on our contract.
\end{itemize}

\noindent After the flash loan is granted, a method on the contract is called
(see the signature in figure \ref{signature}). The most interesting thing that
is passed to the method is passed in the data variable. This block of data is
used to control what the loan is used for. By creating a simple byte code
interpreter, we can control what steps to take. We have created a simple
instruction set with the definition show here, where \code{<data>} is the final
argument to the method\footnote{See figure \ref{signature} for a signature of
the method} \code{initiateFlashLoan}:

\begin{samepage}
\begin{verbatim}
<Uniswap>  ::= 0x00
<Kyber>    ::= 0x01
<exchange> ::= <Uniswap> | <Kyber>

<WETH>   ::= 0x00
<USDC>   ::= 0x01
<DAI>    ::= 0x02
<asset> ::= <WETH> | <USDC> | <DAI>

<instr_header> ::= <exchange>
<instr_data>   ::= <asset> <asset> // Asset to convert from and to

<instruction> ::= <instr_header> <instr_data>

<data> ::= <instruction> <data> | ""
\end{verbatim}
\end{samepage}

\noindent To initiate a flash loan the method \verb|initiateFlashLoan| is
called\footnote{See figure \ref{signature} for a signature of the method}. This
method takes care of setting up the sequence of events to complete the flash
loan. The method calls our \code{callFunction} method and passes the
instructions.

\begin{figure}[H]
\begin{lstlisting}[language=Solidity,numbers=none]
contract Gordon is ICallee, DydxFlashloanBase, Ownable {
    function initiateFlashLoan(
        address _solo,
        address _token,
        uint256 _amount,
        bytes calldata data) { ... }

    function callFunction(
        address sender,
        Account.Info memory account,
        bytes memory data
    ) public override { ... }
}
\end{lstlisting}
    \caption{The contract definition and the key methods' signatures}
    \label{signature}
\end{figure}

\subsection{Opportunity detection}
We have implemented a profit calculator in python which, given the reserves of 2
different exchanges, can calculate how much to borrow and the resulting profits.
This is done with the formula for input, discussed in the arbitrage section.
This is of course not enough to detect opportunities, but would be central in
such a system. We're currently not getting all of the information that would be
required to use this (namely, all of the reserves for all of the exchanges). If
we had this information, it would however be quite simple to detect
opportunities. We would identify all of the duplicate pairs, and run then
through this calculator, and then check if the profits exceed the price of the
loan + the price of the gas used. In terms of complexity, this is a relatively
simple problem, proportional to the number of duplicate pairs (When considering
traditional arbitrage, triangular arbitrage is significantly more complex). This
system would run continuously, and whenever an sufficiently big opportunity
would appear, the contract would be executed.
